\chapter{总结与展望}

\section{总结}

浮游动物图像自动分类识别是浮游动物检测系统中的关键技术,随着浮游动物图像获取技术的不断提高,越来越多的图像数据被采集获取到,仅仅依赖人为的种类鉴别手段已经难以满足对大量数据进行实时快速处理的需要。基于机器学习方法的浮游动物图像分类方法,与其它方法相比,识别准确率更高,因此越来越受到重视。

本文围绕机器学习理论和多特征多分类器组合技术,进行了相关的研究和探索,主要完成了以下工作:

\begin{enumerate}
\item 以浮游生物成像系统和自动识别技术为基础,研究了国内外的发展现状,并且介绍了为什么要进行浮游动物图像分类识别的研究。
\item 对所要分类的浮游动物图像数据集进行了分析,对每类浮游动物的特点做了一个大致的整理,从而能够人为地对这些图像进行分类。
\item 对图像处理领域常用的特征以及在各种分类竞赛中所采用的一些经典特征做了总结,通过理论和实验的结合,从大量的特征中选择适用于浮游动物图像的特征。
\item 针对不同类型的特征,选择适用于该特征的分类算法进行分类器的设计。
\item 提出了一种基于多特征多分类器组合的浮游动物图像分类方法,从而能够综合不同分类器的信息,提高分类准确率。
\end{enumerate}

\section{展望}

尽管本文的算法在浮游动物的识别准确率上有所提高,并且对特征和分类器都进行了总结,做了大量的实验来验证哪些特征和分类器适用于解决浮游动物图像分类这类问题。但是还存在着一些不足,需要做进一步的研究:

\begin{enumerate}
\item 我们的算法是基于Matlab语言的,但Matlab在计算速度上仍然不够快,将来希望用C/C++语言进行编程,进一步提高算法的运行速度。
\item 本文采用的是对分类结果进行融合的方法,还有很多特征融合的方法没有进行尝试,下一步会针对特征融合进行研究,看看对识别效果有没有改善。
\item 每个类别图像数目相差较多时,识别准确率较低,当每个类别数目大致相同时,识别准确率会有所提高,后续会对这一现象继续研究,从理论和实验两方面来解释说明。
\item 由于我们的实验都是基于同一数据集的,可能会带有数据集偏见,今后的研究希望能不依赖于数据集,要找到在各种浮游动物图像数据集上都能取得较好效果的特征和分类器。
\end{enumerate}