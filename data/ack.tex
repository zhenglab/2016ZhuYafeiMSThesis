%%% Local Variables:
%%% mode: latex
%%% TeX-master: "../main"
%%% End:

\begin{ack}
时光匆匆如流水,转眼便又是毕业时节了,在我三年的硕士研究生生涯里,我所收获的不仅仅是愈加丰厚的知识,更重要的是思维方式和表达能力的转变与提升。我很庆幸在这三年我遇到了很多的良师益友,无论在学习上、生活上还是工作上,都给予了我无私的帮助和悉心的照顾,让我在一个温馨、有爱的环境中度过了我的三年研究生生活。

首先,我要感谢我的导师郑海永,这三年来给我最多教育和帮助的人!从我开始读研开始,郑老师一直对我严格要求,在他的指导下,我不仅树立了远大的学习目标,掌握了基本的研究方法,还明白了很多为人处世的道理。本论文从选题到最终完成,每一步都得到了郑老师的帮助和指点,在我迷茫时给予适时的鼓励,帮助我开拓研究思路,解决问题。感谢师母冯丽颖的关怀和教育,使我认识到自身的不足,从而努力提高自己,做一个坚强、会生活、懂生活的新时代女性。在此,谨向导师和师母表示崇高的敬意和衷心的感谢!

感谢实验室所有兄弟姐妹,没有你们的陪伴,我不会拥有这么美好、难忘的三年回忆。感谢我的师姐赵红苗、师妹王如晨,和我并肩作战,在我的研究课题上给予了很多的支持和帮助,让我有了更大提高。

感谢我的家人,谢谢你们无私的付出,在我身后坚定地支持我、关心我、帮助我,有你们在,我更安心!

感谢国家自然科学基金青年科学基金项目“基于视觉注意结合生物形态特征的海洋浮游植物显微图像分析”(批准号:61301240)、国家自然科学基金项目“基于生物形态特征的中国海常见有害赤潮藻显微图像识别”(批准号:61271406)和中央高校基本科研业务费“海洋浮游动物原位探测与分析系统”(批准号:201562023)的资助。

最后,感谢所有关心和帮助过我的人,祝愿你们幸福安康!
\end{ack}
