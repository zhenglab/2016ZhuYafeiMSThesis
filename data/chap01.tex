\chapter{绪论}

\section{课题的研究背景及意义}

地球上海洋的面积为3.6亿平方千米,约占地球表面总面积的71\%,海洋是地球上综合生产力最大的生态系统,对于经济的快速发展起到至关重要的作用。而在海洋当中,海洋浮游生物是数量最多、规模最庞大的生物种类,其主要包括浮游植物和浮游动物两大类。这些浮游生物,绝大多数个体较小(一般从几微米到几毫米),需要借助显微镜才能看清楚它们的身体构造,并且它们缺乏发达的运动器官,游动能力很弱。其中浮游植物属于初级生产者,是食物链的第一个环节,其数量变动对浮游植物群落结构和海洋生态环境的维持起着十分重要的作用。而浮游动物属于次级生产者,它的作用主要表现为:第一,它相当于在浮游植物和其它捕食者之间搭起一座桥梁,使能量可以顺利地由低向高传输,所以其在维持整个海洋中生物(包括浮游植物和鱼类等)数量方面起着举足轻重的作用。第二,浮游动物自身的尸体和摄食产生的粪便中含有有机碳元素,并且以这种形式将N、C、P等元素传输到深海,以丰富海洋的有机物元素。第三,栖息环境的变换会很大程度上地影响浮游动物,表现为群落的结构和和功能参数上的差异,而这些差异可以反应气候变化及海洋环境变化。第四,某些种类的浮游动物在较深水层大量密集,会形成深海散射层,阻碍或干扰水下声纳的信号传播,使声纳失效,因而,它们在国防军事中的作用也越来越受到人们的重视。

浮游生物监测的重点是调查其种类组成和数量分布,其中最重要的监测指标是“优势种类判定”及“密度计算”。传统的浮游生物监测主要是由工作人员在显微镜下观测样本,并对样本进行分类和计数。但是这种监测手段对浮游生物工作者的专业性要求较高,且需要大量人力、财力的支撑,识别速度较慢,无法达到实时分析的需求。在过去的几十年间,浮游生物成像系统得到了快速发展,从而可以在更精细的时间、空间尺度上对浮游生物进行研究。与此同时,为了有效解决由于浮游生物分类学人才稀缺而造成的种类鉴别滞后现象,越来越多的科研工作者投入到利用图像处理和模式识别技术进行浮游动物的自动分类与计数这一研究中,并且在目前阶段已经取得显著进展。

\section{浮游生物图像自动识别方法的国内外研究现状}

浮游生物图像自动识别技术的发展与成像系统的研究和改进密不可分,以下具体介绍这两类技术近年来在国内外的研究现状。

\subsection{浮游生物成像系统的发展}

传统的浮游生物图像通常是经过海洋调查采水取样,然后用显微镜对样本进行拍照而得。由于近几年摄像技术的不断提高和日益增长的海洋浮游生物实时监测的需求,使得新的浮游生物现场及光学成像系统不断涌现。

美国伍兹霍尔海洋研究所(WHOI)的科研团队研制的视频浮游生物录像机(Video Plankton Recorder,VPR)~\cite{davis2005three}可以近距离拍摄到浮游生物的影像,为研究浮游动物提供充足的样本;美国南佛罗里达大学研发的灰度图像颗粒探测系统(Shadowed Image Particle Platform and Evaluation Recorder,SIPPER)~\cite{remsen2004you}采用高速线扫描相机可对水下浮游生物连续成像;法国研制的水下颗粒物和浮游动物图像原位采集系统(Underwater Video Profiler,UVP)~\cite{davis1992video}采用传统的照明设备和经电脑处理的光学技术,可实现浮游动物和颗粒物的剖面观测以及浮游动物图像和颗粒物图像的原位采集;由美国流体成像技术公司研发生产的流式细胞摄像系统(FlowCam)~\cite{benfield2007rapid}能够从流体中快速检测出有机和无机悬浮体,具有连续成像拍照和流式计数功能。近年来,全息成像技术也逐步成为获取浮游生物图像的一种重要手段。英国数家科研机构联合开发一种新的全息成像样机(HOLOMAR)~\cite{Watson2004HoloMar},这款试验机是基于离轴技术,可以获得水下浮游生物图像,在水下$10^5cm^3$的容量内都能精确记录,并且最小分辨率能达到$10\mu m$。美国约翰霍普金斯大学于1999年研发了一款基于水下同轴激光的全息成像系统,在水下$732-1964cm^3$的体积内都能进行勘察,用电荷耦合器件替代了在暗室中进行全息干板记录~\cite{katz1999submersible}~\cite{pfitsch2005development}。

国内,厦门大学的戴君伟等人~\cite{戴君伟2006海洋赤潮生物图像实时采集系统}为采集海洋藻类生物图像研制了一种新的系统,该系统融合了流式细胞技术和常用的成像采集技术,能够快速获得图像数据。
国家海洋技术中心于连生等人~\cite{于翔2009水下全自动显微成像仪}于2009年研发的“水下全自动显微成像仪”,该仪器可以很好地对浮游生物进行拍照;浙江大学陈耀武、洪炎峰等~\cite{tan2012实时浮游生物图像目标智能识别系统设计}开发的“实时浮游生物图像目标智能识别系统”能够对海洋水体中的目标生物进行实时连续走航大面积探测和图像采集;中国海洋大学于新生、周章国等~\cite{zhou2008design}开展了基于光学成像的浮游生物实时在线监测系统研究。综上,在对海洋浮游生物成像的研究中,不断涌现的采集系统保证了充足的研究资源,使基于图像处理的浮游生物研究更具有发展前景。

\subsection{浮游生物图像自动识别技术的发展}

海洋浮游生物的种类多种多样,对这么多的物种进行分类是一项非常繁重的任务。传统的浮游生物识别方法一般是人工地依据形态、尺寸、纹理、颜色等特征进行分类,再根据医学和化学上的一些知识进行进一步判定。但是这种人工识别方法存在一些明显的缺点:首先是对专业性要求较高,工作量大,效率低,浮游生物工作者在识别过程中易疲劳,从而难以保证海洋浮游生物监测的实时性和准确性。近年来发展了很多用来鉴定浮游生物的新技术,例如光谱法、流式细胞术等,但这些方法大多过程繁琐,严重依赖于浮游生物的生理状态,同时仪器昂贵,费用消耗较多。而目前利用数字图像处理技术进行浮游生物自动分类的方法,与其他方法相比,研究成本较低,可以实现精确的种属识别,因而越来越受到重视。

在对海洋浮游生物进行监测的技术中,数字图像处理这一方法的使用可以追溯到1970年,当时计算机处理能力有限,加上光学成像传感器的灵敏度和分辨率较低,对图像只能进行一些基本的信息获取。例如,利用模式识别方法对显微镜下的浮游动物进行简单的计数、尺寸度量和分类~\cite{jeffries1984automated}。近几年,图像处理技术快速发展,计算机性能不断提升,利用图像处理技术进行浮游生物的分类识别也取得了长足的进展。

二十世纪末期,欧洲自动硅藻识别和分类项目建立了自己的硅藻图像数据库并实现了硅藻的自动判识,其使用的方法包括数字图像处理和模式识别。基于 ADIAC 项目,硅藻显微图像分析取得一系列进展~\cite{hicks2006model}~\cite{dimitrovski2012hierarchical}~\cite{wilkinsondiatom}。此外,Embleton 等~\cite{embleton2003automated}结合计算机图像分析利用多层感知器(MLP)神经 网络方法实现 4 种浮游植物自动计数。Culverhouse 等~\cite{culverhouse2003experts}基于人工神经网络进行 6 种甲藻分类。Blaschko 等~\cite{blaschko2005automatic}采用图像分割、特征提取和模式识 别方法实现 12 类浮游生物自动识别。Davis 和 Hu~\cite{hu2006accurate}利用形状和纹理特征采用神经 网络和支撑向量机进行 6 类浮游生物自动识别及丰度估计。Sosik 和 Olson~\cite{sosik2007automated}通过边缘检测等图像处理后提取尺寸、形状、纹理等特征再采用支撑向量机完成 22 类浮游 植物自动分类和丰度估计。Hense 等~\cite{hense2008use}结合荧光图像进行浮游生物结构分析 (PLASA)。Verikas 等~\cite{verikas2012phase}结合相位一致性圆形目标检测、随机优化目标轮廓确定 以及支撑向量机和随机森林分类方法实现微小原甲藻细胞目标的检测和识别。腰鞭毛虫分类项目中,研究者们利用形状和纹理特征,采用人工神经网络分类算法对不同种类的腰鞭毛虫进行了分类~\cite{culverhouse2003expert}。法国国家科学院的研究者们对浮游动物进行了位置特征、尺寸特征、灰度特征、形状特征、生物统计特征以及其它自定义特征的提取,并提供了7种常见的分类算法供选用,实现了对浮游动物的分类识别,识别准确率达到80\%左右。Ellen等人~\cite{Quantifying2015Ellen}对已有的机器学习方法进行了研究,能够对浮游动物分类问题的算法选择、性能调整等提供指导,从而最大限度地提高分类准确率。

与国外相比,国内在利用数字图像处理技术来实现浮游生物的分类识别上起步较晚,因而不管在在研究时间还是研究深度上都是有很大的距离的,并且大多数项目的研究对象都是针对浮游植物(如藻类)的。王明时等人~\cite{王明时2004显微图像分析技术在赤潮生物识别中的应用}首先使用阈值法和形态学法对已有的赤潮生物显微图像进行处理,在图像中得到单个的浮游生物个体,再对这些个体的灰度和轮廓特征进行提取,最终完成分类识别。汪振兴等人~\cite{汪振兴2007赤潮藻类图像自动识别的研究}在对藻类图像的分类识别当中,提取了纹理和形状两种类型的特征,并采用了人工神经网络分类算法训练出分类器,实现了对赤潮藻的自动识别。骆巧琦等人首先对硅藻样本进行了采集,然后通过双轮廓叠加法对采集到的图像进行了分割,再提取了几何描述特征和形状描述特征,经过人工神经网络算法进行训练,最终得到精确的分类器~\cite{骆巧琦2011基于形状特征的硅藻显微图像自动识别}。魏雅娟等提取了基本的形态特征和纹理特征,采用支持向量机作为分类算法,实现了浮游动物图像的自动识别~\cite{魏雅娟2013暗视场浮游动物图像自动识别方法研究}。

综上所述,目前国内外基于图像技术的浮游生物分析主要在于浮游生物的目标识别和计数,且主要呈现出以下特点:

\begin{enumerate}
\item 就研究对象而言,国内外关于浮游动物分类识别的研究明显少于对浮游植物(如藻类)的识别研究。

\item 就研究范围而言,大多集中在某一个种类或者少量几个类别。

\item 就目标表示的特征提取而言,大多数研究方法选择外部形状特征和纹理特征作为浮游生物特征提取和描述,而多数浮游生物纹理特征并不明显,难以作为分类特征,且对特征没有进行有效研究与分析。

\item 就分类技术而言,逐渐趋于使用机器学习算法作为分类器,但一般使用单一的分类器得到分类结果,而没有考虑不同分类器分类结果的融合。
\end{enumerate}

\section{课题来源}

课题来源:国家自然科学基金青年科学基金项目“基于视觉注意结合生物形态特征的海洋浮游植物显微图像分析”(批准号:61301240)、国家自然科学基金项目“基于生物形态特征的中国海常见有害赤潮藻显微图像识别”(批准号:61271406)和中央高校基本科研业务费“海洋浮游动物原位探测与分析系统”(批准号:201562023)。

\section{论文组织结构安排}

本文总结了国际上现有的浮游动物图像分类方法,对其中基于机器学习的浮游动物图像分类算法进行了深入研究,并提出了基于多特征多分类器组合的浮游动物分类算法。本文的主要安排如下:

第一章为绪论部分,主要对浮游动物图像分类研究的背景、意义及国内外研究现状进行介绍,并对全文的主要安排进行了说明。

第二章介绍浮游动物图像分类的一些基础知识。详细介绍了用来进行分类的数据集以及利用机器学习方法进行浮游动物图像分类的过程,对浮游动物图像分类准确率最高的ZooScan系统进行了大致的描述,最后,说明了如何评价一个分类器性能的好坏。

第三章将图像处理和机器视觉领域一些常用的特征进行整理和总结,通过一定的评判标准进行特征选择,最终选取三类特征作为浮游动物图像的特征提取。

第四章简述了机器学习领域常用的分类技术,以分类准确率作为评判,设计适合各类特征的最优分类器。

第五章针对单一分类器存在分类片面性,提出了多特征多分类器组合的方法,详细介绍了算法的基本原理,并通过在包含13类浮游动物的数据集上进行实验证明了我们的方法的有效性。

第六章对全文的工作进行了总结,指出工作中存在的问题并对以后的工作提出了展望。